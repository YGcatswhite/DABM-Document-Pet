%!TEX program = xelatex
\documentclass[a4paper]{ctexart}

\usepackage{listings} 
\usepackage{geometry}
\usepackage{booktabs}
\usepackage{graphicx}
\usepackage{tabularx}
\usepackage{multirow}
\usepackage{enumitem}
\usepackage[bottom]{footmisc}
\usepackage[table,xcdraw]{xcolor}

\renewcommand{\multirowsetup}{\centering}

\geometry{
    left=23mm,
    right=23mm,
    top=23mm,
    bottom=23mm,
}

\setlength{\parskip}{0.5em}

\title{\Huge PetLover 商业计划书}

\author{
  项目成员:\\
  姬筠刚 191250055(PM)\\
  陈梓俊 191250016\\
  丁炳智 191250024\\
  刘庭烽 191250093\\
}
\date{\today}

\begin{document}

\maketitle

\centerline{\includegraphics[]{logo.png}}

\newpage

\tableofcontents

\newpage

\setlength{\parskip}{1em}

\begin{abstract}
  本项目名为PetLover,是由PetLover小组于2021年秋季学期《需求与商业模式创新》课程大作业中设计的软件产品项目。此文档为商业计划书,主要内容包含项目简介、市场与竞争分析、主要产品介绍、财务分析、运营规划、风险预测及分析方案和团队介绍。
  
  本创业计划书属于商业机密,其所涉及的内容和资料只限于PetLover团队成员使用。收到本创业计划书后,收件人应遵守以下规定:
\begin{enumerate}
  \item 本创业计划书仅限于本次课程作业规定人员使用;
  \item 在没有取得本创业团队的书面同意前,不得将本创业计划书全部和或部分复制、影印、泄露或散布给他人。
\end{enumerate}
\end{abstract}


\section{项目简介}

\subsection{项目背景}
宠物,是当代社会人群必不可少的陪伴,无论是在外打拼的年轻人群体,还是安度晚年的老年人群体,抑或是中年人群体,都对宠物饲养有着或多或少的要求。由此,一些宠物交易市场和宠物日常交流平台应运而生。

经过一段时间的市场调研,我们发现:虽然目前市场上已存在宠物交易平台和宠物成长分享平台,但是能够提供一站式宠物服务(包括从宠物买卖、宠物饲养交流、宠物用品到宠物日常分享等的一系列相关服务)的软件平台少之又少。用户往往需要在平台A进行宠物交易,在平台B进行宠物用品购买,在平台C进行宠物疾病咨询,在平台D分享宠物日常……

为了避免宠物主人在各个平台之间“反复横跳”,我们计划打造一款集宠物(及相关用品)交易、宠物疾病咨询、宠物饲养教程、宠物日常分享四大功能于一身的宠物交流平台——PetLover平台(以下简称“PetLover”或“平台”),希望通过一站式服务真正地为宠物爱好者们提供便利;同时通过与各地线下实体宠物店、宠物医院等进行合作,建立全国各地的宠物交易/问诊线下网络,推进线上、线下双线模式的宠物交易和宠物疾病诊疗服务。

PetLover,帮助宠物爱好者们从头见证宠物宝贝的成长!

\subsection{项目结构总览}

\subsection{产品简介}

\subsection{市场情况}

\subsection{商业模式}

\subsection{财务预测}

\section{市场与竞争分析}

\subsection{市场环境}

\subsection{目标市场与需求分析}

\subsection{市场规模}

\subsection{竞品分析}

\subsection{SWOT分析矩阵}

\section{主要产品介绍}

\subsection{产品描述}

\subsection{功能介绍}

\subsection{使用场景}

\section{财务分析}

\subsection{投融资分析}

\subsection{财务分析}
% Please add the following required packages to your document preamble:
% \usepackage[table,xcdraw]{xcolor}
% If you use beamer only pass "xcolor=table" option, i.e. \documentclass[xcolor=table]{beamer}
\begin{table}[]
  \begin{tabular}{|
  >{\columncolor[HTML]{FFFFFF}}c |
  >{\columncolor[HTML]{FFFFFF}}c |
  >{\columncolor[HTML]{FFFFFF}}c |
  >{\columncolor[HTML]{FFFFFF}}c |
  >{\columncolor[HTML]{FFFFFF}}c |
  >{\columncolor[HTML]{FFFFFF}}c |
  >{\columncolor[HTML]{FFFFFF}}c |}
  \hline
  资产(单位:万元) & 初期  & 第一年末 & 第二年末 & 第三年末 & 第四年末 & 第五年末 \\ \hline
  现金        & 200 & 185  & 317  & 492  & 749  & 1136 \\ \hline
  短期投资      & —   & —    & —    & —    & —    & —    \\ \hline
  预付款项      & —   & 194  & 322  & 500  & 745  & 1010 \\ \hline
  存货        & —   & 6    & 6    & 17   & 20   & 28   \\ \hline
  待摊费用      & —   & —    & —    & 6    & 10   & 14   \\ \hline
  流动资产      & 200 & 385  & 645  & 1015 & 1514 & 2188 \\ \hline
  固定资产总额    & —   & 40   & 32   & 42   & 67   & 84   \\ \hline
  累计折旧      & —   & 8    & 6    & 8    & 14   & 20   \\ \hline
  固定资产净值    & —   & 32   & 25   & 35   & 52   & 62   \\ \hline
  无形资产      & —   & 24   & 40   & 64   & 45   & 116  \\ \hline
  资产合计      & 200 & 441  & 710  & 1114 & 1611 & 2366 \\ \hline
  \end{tabular}
  \end{table}
% Please add the following required packages to your document preamble:
% \usepackage[table,xcdraw]{xcolor}
% If you use beamer only pass "xcolor=table" option, i.e. \documentclass[xcolor=table]{beamer}
\begin{table}[]
  \begin{tabular}{|
  >{\columncolor[HTML]{FFFFFF}}c |
  >{\columncolor[HTML]{FFFFFF}}c |
  >{\columncolor[HTML]{FFFFFF}}c |
  >{\columncolor[HTML]{FFFFFF}}c |
  >{\columncolor[HTML]{FFFFFF}}c |
  >{\columncolor[HTML]{FFFFFF}}c |
  >{\columncolor[HTML]{FFFFFF}}c |}
  \hline
  负债和所有者权益(单位:万元) & 初期  & 第一年末 & 第二年末 & 第三年末 & 第四年末 & 第五年末 \\ \hline
  短期借款            & —   & —    & —    & —    & —    & —    \\ \hline
  应付工资            & —   & 64   & 64   & 96   & 112  & 120  \\ \hline
  应付税金            & —   & 66   & 186  & 355  & 438  & 556  \\ \hline
  其他应付款           & —   & 48   & 40   & 48   & 140  & 209  \\ \hline
  流动负债合计          & —   & 180  & 290  & 500  & 690  & 885  \\ \hline
  长期负债            & 40  & 52   & 53   & 0    & 0    & 0    \\ \hline
  实收资本            & 160 & 160  & 160  & 160  & 160  & 160  \\ \hline
  资本公积            & —   & —    & —    & —    & —    & —    \\ \hline
  盈余公积            & —   & 10   & 28   & 54   & 66   & 84   \\ \hline
  未分配利润           & —   & 41   & 180  & 328  & 741  & 1240 \\ \hline
  负债和所有者权益合计      & 200 & 443  & 711  & 1042 & 1657 & 2369 \\ \hline
  \end{tabular}
  \end{table}
% Please add the following required packages to your document preamble:
% \usepackage[table,xcdraw]{xcolor}
% If you use beamer only pass "xcolor=table" option, i.e. \documentclass[xcolor=table]{beamer}
\begin{table}[]
  \begin{tabular}{|
  >{\columncolor[HTML]{FFFFFF}}c |
  >{\columncolor[HTML]{FFFFFF}}c |
  >{\columncolor[HTML]{FFFFFF}}c |
  >{\columncolor[HTML]{FFFFFF}}c |
  >{\columncolor[HTML]{FFFFFF}}c |
  >{\columncolor[HTML]{FFFFFF}}c |}
  \hline
  收益(单位:万元)    & 第一年 & 第二年 & 第三年 & 第四年  & 第五年  \\ \hline
  一、主营业务收入     & 328 & 600 & 880 & 1120 & 1440 \\ \hline
  减:主营业务成本     & 75  & 85  & 92  & 122  & 136  \\ \hline
  减:主营业务税金即附加  & 16  & 16  & 22  & 24   & 27   \\ \hline
  二、主营业务利润(毛利) & 237 & 499 & 766 & 974  & 1277 \\ \hline
  加:其他业务利润     & 15  & 17  & 25  & 30   & 38   \\ \hline
  减:营业费用       & 64  & 18  & 22  & 25   & 28   \\ \hline
  减:管理费用       & 2   & 2   & 5   & 7    & 9    \\ \hline
  减:财务费用       & 3   & 2   & 2   & 3    & 5    \\ \hline
  三、营业利润       & 183 & 494 & 762 & 969  & 1273 \\ \hline
  加:投资收益       & 0   & 0   & 180 & 180  & 182  \\ \hline
  加:营业外收入      & 0   & 0   & 10  & 24   & 30   \\ \hline
  减:营业外支出      & 16  & 26  & 62  & 77   & 94   \\ \hline
  减:所得税费用      & 67  & 187 & 356 & 438  & 556  \\ \hline
  四、净利润        & 100 & 281 & 534 & 658  & 835  \\ \hline
  \end{tabular}
  \end{table}
\subsection{财务比率分析}

\section{运营规划}

\subsection{产品开发规划}

\subsection{市场策略及业务拓展计划}

\subsection{财务规划}

\subsection{团队管理}

\section{风险预测及解决方案}

\subsection{风险识别}

\subsection{风险防范措施}

\section{团队介绍}

\subsection{队长介绍}

\subsection{队员介绍}

\end{document}