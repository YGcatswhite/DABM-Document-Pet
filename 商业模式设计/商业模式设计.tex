%!TEX program = xelatex
\documentclass[a4paper]{ctexart}

\usepackage{listings} 
\usepackage{geometry}
\usepackage{booktabs}
\usepackage{graphicx}
\usepackage{tabularx}
\usepackage{multirow}
\usepackage{enumitem}
\usepackage[bottom]{footmisc}

\renewcommand{\multirowsetup}{\centering}

\geometry{
    left=23mm,
    right=23mm,
    top=23mm,
    bottom=23mm,
}

\setlength{\parskip}{0.5em}

\title{\Huge PetLover 商业模式设计文档}

\author{
  项目成员:\\
  姬筠刚 191250055(PM)\\
  陈梓俊 191250016\\
  丁炳智 191250024\\
  刘庭烽 191250093\\
}
\date{\today}

\begin{document}

\maketitle

\centerline{\includegraphics[]{logo.png}}

\newpage

\begin{abstract}
  本项目名为PetLover,是由本小组于2021年秋季学期《需求与商业模式创新》课程大作业中设计的软件产品项目。此文档为商业模式设计文档,主要内容包含本项目的商业模式设计有关的六大设计方法,包括客户洞察、构思、视觉化思考、模型构建、讲故事和场景。
\end{abstract}



\tableofcontents

\newpage

\setlength{\parskip}{1em}


\section{度量数值}

本文档共包含了38个要点与18条关联关系。平均要点数量约为4个。要点的联系详见第三部分,每个联系之前有要点位置的标注,例如1a表示关键业务的第a条要点。

\section{商业模式设计}

\subsection{客户洞察}

\subsubsection{客户群体1}
\begin{enumerate}[label=\alph*.]
  \item 看:
  \item 听:
  \item 想与感受:
  \item 说与做:
  \item 痛点:
  \item 收益:
\end{enumerate}


\subsection{构思}

\subsubsection{基本商业构思}

\subsubsection{候选商业模式创意}
\begin{enumerate}[label=\alph*.]
  \item 
\end{enumerate}

\subsubsection{最终确定的商业模式创意}

\subsection{视觉化思考}

\subsubsection{可视化画布}

\subsubsection{可视化分析}

\begin{enumerate}[label=\alph*.]
  \item 
\end{enumerate}

\subsection{模型构建}

\subsubsection{商业模式画布}

\subsubsection{商业模式画布要点}

\subsubsection{要点联系}

\subsubsection{新闻、调研及分析}

\subsubsection{市场潜力预估}

\begin{enumerate}[label=\alph*.]
  \item 
\end{enumerate}

\subsection{讲故事}

\subsubsection{公司视角}

\subsubsection{客户1视角}

\subsubsection{.......}

\subsection{场景}

\subsubsection{客户群体1与产品交互}

\subsubsection{......}

\end{document}